\documentclass{article} 
\usepackage{url, graphicx}
\usepackage[margin=1in]{geometry}
\usepackage{listings}

\title{Final submission}
\author{
Andrew Grant\\
\texttt{amg2215@columbia.edu}
\and 
Anton Igorevich \\
\texttt{ain2108@columbia.edu}
\and
 Somya Vasudevan \\
 \texttt{sv2500@columbia.edu}
}
\date{4/28/2017}

\begin{document}

\maketitle

\section{}
We've provide you with links to a bunch of the required files. If any do not work for some reason, please reach out to us via email. Nonetheless, all files will be in the repository which is located at:  \url{https://github.com/andyg7/Graph-Library} which should almost assuredly work and so you look there too to find some of the documents.
\section{Development Environment}
\begin{itemize}
\item GCC 6.2
\item OS: Ubuntu 16.10
\item C++ standard libary used: c++1z
\item Compiler options: -fconcepts
\end{itemize}

\section{Links to all requires files}
\begin{itemize}
\item Repository of project: \url{https://github.com/andyg7/Graph-Library}
\item Source code of graph library: \url{https://github.com/andyg7/Graph-Library/tree/master/src}
\item Tests: \url{https://github.com/andyg7/Graph-Library/tree/master/tests}
\item Examples of using library:
\begin{itemize}
\item \url{https://github.com/andyg7/Graph-Library/tree/master/cities_examples}
\item \url{https://github.com/andyg7/Graph-Library/tree/master/examples}
\item \url{https://github.com/andyg7/Graph-Library/tree/master/expander_examples}
\end{itemize}
\item Tutorial: \url{https://github.com/andyg7/Graph-Library/blob/master/docs/Tutorial.pdf}
\item Design Document: \url{https://github.com/andyg7/Graph-Library/blob/master/docs/DesignDocument.pdf}
\item Third Party code - we used some concepts from \url{https://github.com/CaseyCarter/cmcstl2}
\item Commit history: \url{https://github.com/andyg7/Graph-Library/commits/master}
\item Real usage of library - as solved the 8 puzzle game using our library. The code to do this is here:  \url{https://github.com/andyg7/Graph-Library/tree/master/expander_examples}
\end{itemize}




\end{document} 